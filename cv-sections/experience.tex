%----------------------------------------------------------------------------------------
%	SECTION TITLE
%----------------------------------------------------------------------------------------

\cvsection{Experience}

%----------------------------------------------------------------------------------------
%	SECTION CONTENT
%----------------------------------------------------------------------------------------

\begin{cventries}

%------------------------------------------------

\phantomsection\label{dsoFulltime}

\cventry
{
    \textbf{Member of Techincal Staff: Robot Perception Engineer}
} % Job title
{DSO National Laboratories} % Organization
{Singapore} % Location
{Aug. 2023 - Present} % Date(s)
{ % Description(s) of tasks/responsibilities
    \begin{cvitems}
        \item { LiDAR perception engineer focusing on developing novel based algorithms (in C++) for real-world autonomous robots in challenging environments. Currently building experience with classical and deep learning-based methods for a broad spectrum of tasks like semantic segmentation, localization and mapping, and pose-graph optimization. }
    \end{cvitems}
}


%------------------------------------------------

\phantomsection\label{radarSLAM}

\cventry
{
    \textbf{Research Intern: 3D Radar Odometry for adverse weather conditions} 
     \githubButton{CFEAR2D}
} % Job title
{DSO National Laboratories} % Organization
{Singapore} % Location
{Jun. 2022 - Aug. 2022} % Date(s)
{ % Description(s) of tasks/responsibilities
    \begin{cvitems}
        \item { Successfully implemented 2D version of \fancyhref{https://arxiv.org/pdf/2105.01457.pdf}{CFEAR radar odometry paper by Adolfsson et. al.} in C++, using the OpenCV and Ceres Solver libraries.}
        \item { Ported 2D code to 3D, for use with a new 3D radar. Algorithm to be adapted and used for novel 3D radar odometry research. }
    \end{cvitems}
}


%------------------------------------------------

\phantomsection\label{radarOdometry}

\cventry
{
	\textbf{Research Intern:} Real-time radar odometry for adverse weather conditions using phase correlation and local pose-graph estimation
} % Job title
{DSO National Laboratories} % Organization
{Singapore} % Location
{Jun. 2020 - Aug. 2020} % Date(s)
{ % Description(s) of tasks/responsibilities
	\begin{cvitems}
		\item { Successfully implemented phase correlation and partially implemented local pose-graph estimation components of the \fancyhref{https://irap.kaist.ac.kr/publications/yspark-2020-icra.pdf}{PhaRaO radar odometry paper by Park et. al.} in C++, using the OpenCV and Ceres Solver libraries.}
		\item { Algorithm to be adapted and actively used for organisation's unmanned ground vehicles. It will be part of a radar odometry pipeline, to supplement LiDAR for navigation in adverse conditions such as rain and dust. }
	\end{cvitems}
}

%------------------------------------------------

\phantomsection\label{computerVisionProjectCMU}

\cventry
{
	\textbf{Research Intern:}  Evaluating multi-view human pose estimation algorithm on CMU Panoptic Studio and other datasets \infoButton{https://samleo8.github.io/projects/\#evaluating-multi-view-human-pose-estimation-algorithm-on-cmu-panoptic-studio-and-other-datasets-} \githubButton{learnable-triangulation-pytorch}
} % Job title
{CMU Human And Robotic Partners (HARP) Lab} % Organization
{Singapore} % Location
{Nov. 2019 - May 2019} % Date(s)
{ % Description(s) of tasks/responsibilities
	\begin{cvitems}
		\item { Briefly evaluated various state-of-the-art methods for multi-view 3D human pose estimation, and sought to adapt the most suitable one for use on a dataset which the lab had collected prior.}
		\item { Successfully developed an open source toolkit in Python for evaluating the \fancyhref{http://domedb.perception.cs.cmu.edu}{CMU Panoptic Dataset} using Iskakov et. al.'s  \fancyhref{https://github.com/karfly/learnable-triangulation-pytorch}{learnable triangulation} algorithm.}
		\item { Also worked on generalising the toolkit for use with general datasets, including that of the lab.}
        \item { Been approached by PhD students to help integrate my work into their active research. }
	\end{cvitems}
}

%------------------------------------------------

\phantomsection\label{computerVisionProjectDSO}

\cventry
{
	\textbf{Research Intern:} Integrated Data Annotation and Augmentation Tool for Object Recognition and Tracking
} % Job title
{DSO National Laboratories} % Organization
{Singapore} % Location
{Feb. 2019 - Apr. 2019} % Date(s)
{ % Description(s) of tasks/responsibilities
	\begin{cvitems}
		\item { Successfully developed a data annotation and augmentation tool in C\#. The tool was integrated with a proprietary algorithm provided by our mentor (adapted from YOLOv2 and another proprietary tracking algorithm).}
		\item { Used the tool we developed to generate bounding box data, correct it manually, and augment it automatically. We then used the data for retraining the said algorithm.}
		\item { Also explored ways to improve the algorithm by adapting it for use with YOLOv3 and other trackers.}
	\end{cvitems}
}

%------------------------------------------------

\phantomsection\label{wirelessSoundProcessing}

\cventry
{\textbf{Research Intern:} Low-powered Wireless Sound Processing
} % Job title
{DSO National Laboratories} % Organization
{Singapore} % Location
{Jan. 2017 - Feb. 2017} % Date(s)
{ % Description(s) of tasks/responsibilities
\begin{cvitems}
	\item { Successfully implemented and tested algorithm for communication between a TI-MSP430 microcontroller and an ASIC Chip \textit{(Application Specific Integrated Circuit)}, via the Serial Peripheral Interface \textit{(SPI)} Protocol. }
	\item { Implemented data transmission from said microcontroller to another via Wi-Fi, to allow for wireless data processing. }
	\item { Algorithm further modified by organisation for their internal applications. }
\end{cvitems}
}

%------------------------------------------------

\phantomsection\label{lightDirectionFinderProject}

\cventry
{\textbf{Research Intern:} Optically-Illuminated Directional Sensing for Guidance Systems \infoButton{https://samleo8.github.io/projects/\#optically-illuminated-directional-sensing-for-guidance-and-alignment-systems}
} % Job title
{DSO National Laboratories} % Organization
{Singapore} % Location
{Apr. 2015 - Mar. 2016} % Date(s)
{ % Description(s) of tasks/responsibilities
\begin{cvitems}
	\item { Successfully prototyped an analog circuit capable of demodulating and amplifying a frequency-modulated laser signal. }
	\item { Programmed algorithm on TI-MSP430 Launchpad microcontroller to digitise analog input from circuit. \newline Digitised signal then used to sense direction of laser-point, and actuate a novel omni-directional land robot.}
	\item { Represented Singapore at Intel International Science and Engineering Fair (ISEF). }
\end{cvitems}
}

%------------------------------------------------

\phantomsection\label{eyeTrackingProject}

\cventry
{\textbf{Research (Team):} Analysis of Multimodal Interaction methods for Multi-Tasking \infoButton{https://samleo8.github.io/projects/index.html\#analysis-of-multimodal-interactions-for-simultaneous-spatial-and-cognitive-tasks}
} % Job title
{DSO National Laboratories} % Organization
{Singapore} % Location
{Apr. 2014 - Jan. 2015} % Date(s)
{ % Description(s) of tasks/responsibilities
	\begin{cvitems}
		\item { Tested intuitiveness and efficiency of multiple interaction methods (eye-tracking, gestures, touch, speech and keyboard) in completing load-intensive tasks, via a \fancyhref{https://samleo8.github.io/games/playFlash.html?name=FocusFireGame}{custom-designed Flash game.}}
		\item { Helped team design said Flash game, and a \fancyhref{https://samleo8.github.io/eye-tracking-website/}{custom website} to highlight advantages of eye-tracking. }
		\item { Presented to then Minister of State for Defence, Mr Maliki Osman, at the Young Defence Scientists Congress }
	\end{cvitems}
}

%------------------------------------------------

\end{cventries}